\documentclass{book}
\usepackage[english]{babel}
\usepackage{graphicx}
\usepackage{hyperref}
\begin{document}

\title{\TeX4ht\ Template for Online Books}
\author{Michal Hoftich}
\maketitle
\tableofcontents

\chapter{Introduction}

This template is designed to facilitate the export of each book chapter into a
separate HTML page, making it ideal for publishing books, documentation, or
lecture notes in a structured and easily navigable web format.

\paragraph{Key Features:}

\begin{itemize}
  \item Modular Structure: Each chapter is converted into an independent HTML
    file, improving load times and organization.

  \item Seamless Navigation: Built-in links (previous/next chapter, table of contents in sidebar) ensure smooth reader navigation.

  \item Customizable Styling: Supports CSS theming to match your project’s design (e.g., academic, technical, or minimalist).
\end{itemize}

\paragraph{Use Cases}

\begin{itemize}
  \item E-books / Online Textbooks – Publish course materials with chapter-based access.

  \item Technical Documentation – Break long manuals into searchable sections.

  \item Collaborative Projects – Combine multiple authors’ work into a unified web book.
\end{itemize}

\paragraph{How It Works}

The template uses \href{https://tug.org/tex4ht/}{TeX4ht}, a LaTeX-to-HTML converter, to automate the
splitting of .tex files while preserving cross-references, footnotes, and
bibliographies. 

\paragraph{Automation Features}
\begin{itemize}
    \item \textbf{GitHub Actions Integration}: 
    \begin{itemize}
        \item Automatic compilation triggered on every commit/push to the repository
        \item Pre-configured workflow handles LaTeX → HTML conversion via \TeX4ht
        \item Build status monitoring and error reporting
    \end{itemize}
    
    \item \textbf{GitHub Pages Publishing}:
    \begin{itemize}
        \item Automatically deploys the compiled HTML output to GitHub Pages
        \item Creates a publicly accessible website (\texttt{https://yourname.github.io/repository-name})
        \item Supports custom domains if needed
    \end{itemize}
    
    \item \textbf{Continuous Deployment}:
    \begin{itemize}
        \item Ensures the online version always matches the repository content
    \end{itemize}
\end{itemize}

\chapter{Basic Usage}

To start your own online book using this template, click the "Use this
template" button at the top of the
\href{git@github.com:michal-h21/tex4ht-booksite.git}{GitHub repository page} (see figure~\ref{fig:template}).
This will create a new repository in your account with the same structure and
files. You can then clone it, customize the content, and start building your
own book.

\begin{figure}[htbp]
  \includegraphics[width=\textwidth,alt={Image of button for use of this repository as a template}]{img/template-use.png}
\caption{Using the template to create a new repository}
\label{fig:template}
\end{figure}

Your book's core content goes in \texttt{document.tex} (this file currently contains
sample text that you'll replace with your own work).

For larger projects, we recommend splitting content into separate files (one
per chapter, for example) and including them using \verb|\include{filename}| commands.

Remember:

\begin{itemize}
  \item Keep document.tex as your master file

  \item The build process relies on this structure

  \item All \verb|\include| paths should be relative to \texttt{document.tex}
\end{itemize}


\chapter{Compiling the Book}


You can compile the output files using any standard \LaTeX\ distribution which includes \href{https://www.tug.org/tex4ht/}{\TeX4ht}.

To compile the PDF version use Lua\LaTeX:

\begin{verbatim}
$ lualatex document.tex
\end{verbatim}

For the HTML version, you can use \href{https://github.com/michal-h21/make4ht}{make4ht}, build system for \TeX4ht.

\begin{verbatim}
$ make4ht -e build.lua -c config.cfg -lj index -d out document.tex
\end{verbatim}

Since we need both a Lua build script (for make4ht) and a \TeX4ht config file, this command ends up being a bit more involved.

The \verb|-e build.lua| option loads the build file. It modifies the output fil
The \verb|-l| option  ensures that Lua\LaTeX\ is used as the compiler. 



\chapter{Automated HTML Output}

This chapter explains how \href{https://docs.github.com/en/actions/writing-workflows/quickstart}{GitHub Actions}
is used to automatically generate and publish an HTML version of the handout whenever changes are pushed to the \texttt{main} branch.

The output is built using \texttt{make4ht} and published to the \texttt{gh-pages} branch,
making it easy to share a web-readable version of the talk.

Key parts of the workflow that builds and publishes the HTML:

\begin{verbatim}
- name: Run make4ht
  uses: xu-cheng/texlive-action/full@v1
  with:
    run: |
      make4ht -lj index -a debug -d out book.tex

- name: Publish the web pages
  uses: peaceiris/actions-gh-pages@v3
  with:
    github_token: ${{ secrets.GITHUB_TOKEN }}
    publish_dir: ./out
\end{verbatim}



The workflow is defined in the \texttt{.github/workflows/main.yml} file.
You can edit this file to customize the build process, such as changing the options passed to \texttt{make4ht}.

It uses two GitHub Actions: \href{https://github.com/xu-cheng/texlive-action}{xu-cheng/texlive-action}
and \href{https://github.com/peaceiris/actions-gh-pages}{peaceiris/actions-gh-pages}.
The first one lets you use any command available in the TeX Live installation, such as \texttt{make4ht} or \texttt{lualatex}.
The second one publishes the contents of a specified directory to the \texttt{gh-pages} branch of your repository,
which is used by GitHub Pages to serve static content.



\section{Automatic HTML Build}
Changes pushed to \texttt{main} branch trigger a GitHub Actions workflow that:

\begin{itemize}
  \item Compiles \texttt{handout.tex} to HTML using \texttt{make4ht}
  \item Publishes the output to the \texttt{gh-pages} branch
\end{itemize}

The following command is used for the compilation:

\begin{verbatim}
make4ht -lj index -a debug -d out handout.tex
\end{verbatim}


This builds the HTML into the \texttt{out/} folder, which is then published
using the \texttt{peaceiris/actions-gh-pages} action, specified by the
\texttt{publish\_dir} setting.


\paragraph{Why \texttt{-j index}?}
\begin{itemize}
  \item The \texttt{-lj index} option is a shorthand for \texttt{-l -j index}
  \item The \texttt{-j index} option sets the HTML output filename to \texttt{index.html}
  \item This lets you use clean URLs like:
  
\begin{verbatim}
https://username.github.io/repo/
\end{verbatim}

\end{itemize}


There’s no need to specify the filename in the link — GitHub Pages
automatically looks for \texttt{index.html} by default. This makes it easier to share
the presentation and helps avoid broken links due to filename mismatches.

For example, this book is available at: \url{https://michal-h21.github.io/tex4ht-booksite/}.

\chapter{Github Pages}

\section{Github Actions Interface}
\includegraphics[width=\textwidth]{img/github-actions.png}

After you push changes to the \texttt{main} branch, you can check the \texttt{Actions} tab in your
Github repository. It shows the status of the workflow, including whether it ran successfully or if there were any errors.


\section{Errors}
\includegraphics[width=\textwidth]{img/github-error.png}


You can also check the logs of the workflow run to see what went wrong.
If you encounter an error, it will be displayed in the logs, and you can use that
information to debug the issue.

In this case, the filename of the TeX file was incorrect. I had to fix the filename in the GitHub Actions YAML file.


\section{Setup Github Pages}
\includegraphics[width=\textwidth]{img/github-pages.png}

Once the workflow runs successfully, you can set up GitHub Pages to serve the \texttt{gh-pages} branch.

All output files produced by \texttt{make4ht} will be served on the web.
They will be available at:
\verb|https://username.github.io/repo/|,
where \texttt{username} is your GitHub username and \texttt{repo} is the name of your repository.
\end{document}
